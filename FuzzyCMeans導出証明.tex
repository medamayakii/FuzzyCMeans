\documentclass[uplatex,dvipdfmx,10pt]{beamer}
\usepackage{amsmath,amssymb}
\usepackage{comment}
\usepackage{bxdpx-beamer}% dvipdfmxなので必要
\usetheme{Singapore}
\usefonttheme[onlymath]{serif}

\begin{document}
\section{FCM法の更新則の導出(補足)}
\begin{frame}

ラグランジュ乗数を$\lambda_{i}$, s.t. $i=1, \cdots, n$とする.
ラグランジュ関数を$L$とおくと,
\[
L=\sum_{c=1}^{C} \sum_{i=1}^{n}\left(u_{c i}\right)^{\theta} d_{c i}+\sum_{c=1}^{C} \lambda_{i}\left(\sum_{c=1}^{c} u_{c i}-1\right)
\]
最適性の必要条件は,
\[
\frac{\partial L}{\partial u_{c i}}=\theta\left(u_{c i}\right)^{\theta-1} d_{ci}+\lambda_{i}=0
\]
$x_i\ne b_c$のとき
\begin{equation}
u _{c i}=\left(\frac{-\lambda_{i}}{\theta d_{c i}}\right)^{\frac{1}{\theta-1}}
\end{equation}

\begin{equation}
\left(\frac{-\lambda_i}{ \theta}\right)^{\frac{1}{\theta-1}}=u_{c i}(d_{ci})^{\frac{1}{\theta-1}}
\end{equation}

\end{frame}
\begin{frame}

$\sum_{c=1}^{C} u_{c i}=1$が成り立つので
\begin{equation}
\sum_{c=1}^{C}\left(\frac{-\lambda_{i}}{\theta d _{c i}}\right)^{\frac{1}{\theta-1}}=1
\end{equation}
(3)において$c$を$l$とすると,
\begin{equation}
\sum_{l=1}^{C}\left(\frac{-\lambda_{i}}{\theta d_{l i}}\right)^{\frac{1}{\theta-1}}=1
\end{equation}
(2)(4)より,
\[
\sum_{\ell=1}^{C} u_{c i}\left(\frac{d_{c i}}{d _{l i}}\right)^{\frac{1}{\theta-1}}=1
\]
\[
u_{c i}=\left(\sum_{l=1}^{C}\left(\frac{d_{c i}}{d _{li}}\right)^{\frac{1}{\theta-1}}\right)^{-1}
\]
\begin{equation}
u_{ci}=\left(\sum_{l=1}^{C}\left(\frac{\left\|x_{i}-b_{c}\right\|^{2}}{\left\|x_{i}-b_{l}\right\|^{2}}\right)^{\frac{1}{\theta - 1}}\right)^{-1}
\end{equation}
\begin{flushright}
  証明終
\end{flushright}
\end{frame}

\begin{frame}

\begin{equation}
\begin{aligned}
J_{f c m} &=\sum_{c=1}^{c} \sum_{i=1}^{n} u_{c i}^{\theta} d_{c i} \\
&=\sum_{c=1}^C \sum_{i=1}^{n} u_{c i}^{\theta}\left\|x_{i}-b _c\right\|^{2}
\end{aligned}
\end{equation}
最適性の必要条件は
\begin{equation}
\frac{\partial}{\partial b_c} \left(\sum_{i=1}^{n} u_{c i}^{\theta}\left\|x_{i}-b _c\right\|^{2}\right)=\sum_{i=1}^{n} u_{c i}^{\theta}(-2)\left(x_{i}-b _c\right)=0
\end{equation}

\begin{equation}
\sum_{i=1}^{n} u_{c i}^{\theta} x_{i}=\sum_{i=1}^{n} u_{c i}^{\theta} b_{c}
\end{equation}
から
\begin{equation}
b_{c}=\dfrac{\displaystyle \sum_{i=1}^{n} u_{c i}^{\theta} x_{i}}{\displaystyle \sum_{i=1}^{n} u_{c i}^{\theta}}
\end{equation}
を得る.
\end{frame}

\begin{comment}

\section{FCM法とkmeans法の局所解頻度の比較}
データセットとして,winequality-red(インスタンス数:1599,パラメータ数:11)を用いた.

1000回試行したところ,FCM法では局所解は発生しなかった.
kmeans法では,14種類の局所解が得られその頻度は,2,46,285,25,22,14,252,112,9,1,31,5,2,194回ずつ起こった.

\begin{center}
\begin{verbatim}
  Input variables (based on physicochemical tests):
  1 - fixed acidity
  2 - volatile acidity
  3 - citric acid
  4 - residual sugar
  5 - chlorides
  6 - free sulfur dioxide
  7 - total sulfur dioxide
  8 - density
  9 - pH
  10 - sulphates
  11 - alcohol
  Output variable (based on sensory data):
  12 - quality (score between 0 and 10)
\end{verbatim}
\end{center}
\end{comment}

\end{document}
